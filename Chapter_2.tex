\newpage
\textbf{3 Исследование и анализ вызванной динамики сигналов}
\refstepcounter{chapter}


\addcontentsline{toc}{chapter}{3 Научно-исследовательская часть}

\section{Биотехническая система для роботизированной
ультразвуковой визуализации}
\begin{figure}[!h]
\begin{center}
\includegraphics[width=\textwidth]{БТС.pdf}
\caption{\centering \onehalfspacing {Структурная схема БТС объективного контроля слухового восприятия спектрофотометрическим методом}}
\label{БТС}
\end{center}
\end{figure}

Для разработки БТС для неинвазивного измерения параметров гемодинамики височных долей коры головного мозга с помощью спектрофотометрического метода при одновременной подаче звуковых сигналов необходимо выполнить следующие задачи:

– определение необходимых технических требований к конструкции БТС,

– моделирование всех блоков системы,

– выбор комплектующих всех составных частей каждого блока,

– разработка конструкторской документации

– анализ технологичности конструкции изделия.

\subsection{Требования к конструкции БТС}
Основными задачами данной разработки является повышение точности позиционирования оптических датчиков при билатеральных измерениях (слева и справа) за счёт проектируемой конструкции и объединение устройства подачи звукового сигнала со спектрофотометрическими датчиками. Исходя из этого, сформулированы следующие требования к конструкции БТС:

– конструкция должна обеспечивать надежную фиксацию датчиков на голове пациента, их точное позиционирование,

– изделие должно быть устойчиво к двигательным артефактам,

– изделие должно обеспечивать возможность регулирования параметров таких параметров, как частота и амплитуда звукового сигнала, включение/выключение нужных каналов,

– фиксация датчиков на голове пациента не должна приводить к повреждающему механическому воздействию,

– изделие должно быть устойчиво по отношению к любым методам стерилизации или дезинфекции,

– масса изделия до 1 кг (без беспроводного блока передачи данных),

– обеспечение шумоподавления.

Технические требования:

– входное напряжение 7,5 В,

– непрерывный режим работы источников,

– 4 канала,

– мощность не более 5 мВт.


\subsection{Описание конструкции блоков}
Конструкция состоит из 5 блоков: блока подачи аудиосигнала (наушники), датчиков fNIRS, блока питания, блока обработки сигнала, Bluetooth-модуля. Конструкторская разработка проводилась для первых 2-х блоков.

В качестве блока подачи аудиосигнала могут выступать аудиометрические наушники с хорошей шумоизоляцией. В соответствии с вышеизложенными задачами, конструктивно он должен состоять из следующих частей: дуги наушников, подвижной части корпуса (вращательное движение), накладок и составного динамика. Для обеспечения удобной фиксации на голове пациента в дугу наушников встроены рейки, позволяющие регулировать положение накладок по высоте. Рейки закреплены с помощью реечного фиксатора. Нижний корпус вращается с помощью узла вращения, что обеспечивает плотное прилегание накладок к ушам. На рисунке \ref{ushi} показано взаимное расположение элементов данного блока.

\begin{figure}[!h]
\begin{center}
\includegraphics[width=0.5\textwidth]{12.pdf}
\caption{\centering Элементы блока подачи аудиосигнала}
\label{ushi}
\end{center}
\end{figure}

Два симметрично расположенных датчика должны покрывать височную долю коры головного мозга, длина которой около 70 мм, а ширина 30 мм \cite{litlink11}. Слуховые центры расположены в верхней височной извилине, которая находится примерно на 2 см выше уха. Конструкция должна обеспечить точное проецирование поверхности датчика на слуховую кору. Геометрическое расположение четырех световых излучателей и детектора на модуле показано на рисунке \ref{vis}. Детектор расположен в центре устройства, четыре светодиода расположены в противоположных углах на равных расстояниях 38 мм от детектора, которые закреплены с помощью стальных шарнирных штифтов диаметром 2 мм. Для оптимального качества сигнала важно, чтобы детектор света и излучатели находились как можно ближе к коже головы, но в то же время были перпендикулярны поверхности для обеспечения максимальной чувствительности и проникновения света. Чтобы  минимизировать влияние на сигнал из-за перемещения головы, конструкция светодиодов должна включать в себя пружины. Светодиоды не жестко соединены с корпусом датчика, а встроены в подвижные держатели и могут вращаться вокруг оси. Пружина прижимает светодиод к поверхности головы, тем самым обеспечивая предотвращая потерю контакта во время движения. Поворотное соединение удерживают светодиод перпендикулярно к поверхности, обеспечивая при этом небольшие отклонения для удобства. Подпружиненные светодиодные держатели построены с использованием 20 длинных акриловых стеклянных трубок с наружным диаметром 6 мм, длинных алюминиевых трубок с наружным диаметром 7 мм и толщиной стенки 0,5 мм и 18 мм-длинных металлических пружин с диаметром 4 мм. Многоволновой светодиод припаян к 3-проводному ленточному кабелю, который затем проходит через стеклянную трубку и металлическую пружину. Также светодиод заключен в толстую непрозрачную трубку из резины для предотвращения рассеянного света и амортизации. Для уменьшения влияния рассеянного света и для амортизирующих целей детектор заключен в непрозрачную резиновую трубку. Кроме того, корпус датчика fNIRS может быть окрашен непрозрачной черной краской с внутренней стороны для минимизации световых эффектов на датчик с направлений, отличных от перпендикулярных к чувствительной поверхности. 

Конструктивное решение закрепления датчиков на корпусе наушников – листовая пружина, закрепленная на винтах в нижнем корпусе и на корпусе датчика, что позволяет за счёт изгиба держать нагрузку и давить вперёд в сторону головы (рис. \ref{пруж}). Крепление корпуса датчика к стойке осуществляется за счет стальных шарнирных штифтов диаметром 4 мм (рис. \ref{соед}).

Проанализировав из каких составных элементов состоит каждый блок, составим список покупных изделий:\\
Покупные элементы блока:

– светодиоды (4 шт.),

– детектор,

– 8 винтов А.M3-6gx5,

– акустический кабель AW208.

\begin{figure}[!h]
\begin{center}
\includegraphics[width=0.5\textwidth]{соед.png}
\caption{\centering Закрепление корпуса датчика к листовой стойке}
\label{соед}
\end{center}
\end{figure}
В качестве излучаещего устройства можно использовать только лазерные диоды (ЛД) и светоизлучающие диоды (светодиоды) \cite{litlink15}. ЛД имеют высокую интенсивность излучения, могут быть использованы для получения высококачественного сигнала fNIRS. С другой стороны, их недостатки (перегрев, более высокие требования к безопасности, высокая стоимость и ограниченная длина волны) не позволяют использовать их в портативном устройстве. Поскольку светодиоды малы, сравнительно дешевы, имеют большой разброс по длинам волн, а проблемы нагрева менее критичны, они являются более подходящими для данной конструкции. Светодиоды позволяют проводить исследование одновременно с двух длин волн, что необходимо для расчета относительных показателей при непрерывном режиме. В исследовании \cite{litlink23} пришли к выводу, что использование длин волн 830 нм и 690-750 нм является оптимальным решением. Основываясь на анализе трехслойной модели, авторы работы \cite{litlink24} определили, что предпочтительные длины волн - 887 и 704 нм. Усредним эти результаты и выберем светодиод с длинам волн 760 и 850 нм. 

В роли детектора был выбран кремниевый фотодиод для детектирования света в системе fNIRS. Его преимуществами являются: маленький размер, высокий динамический диапазон и скорость работы. Еще одно преимущество заключается в том, что он может соприкасаться непосредственно с поверхностью кожи, что является наиболее эффективным методом. Из-за фиксированной модуляции оптического сигнала необходима полоса пропускания в несколько кГц. С учетом максимальной чувствительности и минимального уровня шума был выбран детектор OPT101. Он представляет собой монолитный фотодиод с одним источником питания и интегрированным трансимпедансным усилителем.

\begin{figure}[!h]
\begin{center}
\includegraphics[width=0.5\textwidth]{височная.pdf}
\caption{\centering \onehalfspacing {Приблизительное расположение источников и детектора на датчике в проекции на височную долю: красные точки - источники, черная - детектор, синие - точки наивысшей чувствительности к функциональной активности коры головного мозга}}
\label{vis}
\end{center}
\end{figure}

\section{Исследование акустических и механических параметров смеси ПВС-ПДМС}
Данное исследование было проведено для изучения возможностей метода К-БИК спектрофотометрии в оценке и контроле слухового восприятия. Целью эксперимента являлось измерение динамических изменений в концентрациях гемоглобина при стимуляции слуховой коры относительного исходных значений (среднее значение в покое) и разработка алгоритма обработки. Так как абсолютные значения параметров для этого не требуются, был выбран непрерывный режим, позволяющий отследить относительные показатели. 

\subsection{Описание и методика эксперимента}
Эксперимент проводился при помощи спектофотометрического прибора – тканевого оксиметра «OxiplexTS» и калибровочных блоков.

OxiplexTS – устройство, позволяющее измерять концентрацию оксигенированного и дезоксигенированного гемоглобина в тканях. Устройство работает, излучая ближний инфракрасный свет (NIR) в ткани на известных расстояниях от детектора. Используется свет двух различных длин волн (692 и 834 нм), который модулируется на радиочастотной частоте 110 МГц \cite{litlink26}. Собранный свет измеряется и обрабатывается, а также определяются коэффициенты поглощения и рассеяния среды. После определения поглощения и рассеяния применяется предположение о том, что гемоглобин является единственным значимым поглотителем, и вычисляются концентрации оксигенированного и деоксигенированного гемоглобина. OxiplexTS использует теорию миграции фотонов через сильно рассеивающие среды. Это позволяет проводить абсолютные измерения поглощения и рассеяния в высоко рассеивающей среде, такой как человеческая ткань.

Устройство измеряет следующие параметры в единицах АЦП (AC - амплитуда модуляции, DC – средний уровень интенсивности, Phase – фазовый сдвиг сигнала в градусах). Чтобы получить затухающие интенсивности с разных расстояний, надо AC или DC умножить на соответствующий калибровочный коэффициент, а к фазе – прибавить соответствующий калибровочный коэффициент. В данном эксперименте важны только относительные показатели поглощения и концентраций хромофоров, поэтому используется непрерывный режим модуляции. Частота дискретизации была выбрана 5 Гц (измерения проводились каждые 0,2 секунды). Так как измерения ведутся в непрерывном режиме, который чаще всего используется для картирования и также используется в конструкторской разработке, то из измеренных данных использовались значения DC (1-8) и рассчитывались сначала относительные изменения коэффициента поглощения для двух длин волн, по которым с учётом процентного содержания воды определяются относительные изменения концентраций окси-, дезокси- и общего гемоглобина. На первое измеренное значение DC(t0) нормировались все остальные измерения. На рисунке показано расположение источников (диодов) разных длин волн и расстояний. 

Протокол измерений был следующим: испытуемый сидит на стуле. Датчик канала А (Oxiplex) расположен с левой стороны над ухом; датчик канала В – с правой стороны над ухом. Точность позиционирования линейного 4-х-дистантного датчика в области проекции слуховой коры на поверхности головы никак не контролировалась. Оба датчика (каналов А и В) калибровались на церебральном блоке с указанными коэффициентами поглощения и транспортного рассеяния для обеих длин волн зондирующего излучения. На голове испытуемого были надеты внешние наушники, звуковой сигнал подавался в оба уха одновременно. Сигнал записывался десять минут, в течение которых чередовались измерения в покое (3 минуты) и измерения во время стимула (2 минуты). В качестве стимуляции слуховой коры был выбран текст аудиокниги. Повторная стимуляция была проведена для того, чтобы исключить сигналы, не связаные с вызванной активностью головного мозга.

\begin{figure}[!h]
\begin{center}
\includegraphics[width=0.5\textwidth]{датчик.png}
\caption{\centering Схема многодистантного оптоволоконного датчика}
\label{дат}
\end{center}
\end{figure}

\subsection{Анализ полученных данных}

\section{Исследование зависимости качества ультразвукового изображения от силы и угла прижатия УЗ-датчика}
\subsection{Показатели качества изображений с ультразвуковых визуализационных систем}
\subsection{Алгоритм работы программы для роботизированной УЗ-визуализации}
\subsection{Описание и методика эксперимента}
\subsection{Составление соответствующих выводов о влиянии механических параметров (силы и угла наклона) на полученное УЗ-изображение}

\section{Выводы к главе 3}
Все зарегистрированные сигналы были импортированы в MATLAB для дальнейшей обработки. На рисунке \ref{45} изображена схема реакции организма на внешний стимул. Можно заметить, что гемодинамические изменения происходят через 2–5 с после начала стимуляции. Этот факт нужно учитывать, анализируя полученные данные.

Расчет изменений концентраций окси, дезокси- и общего гемоглобина проводился в среде MATLAB по следующему алгоритму:

– умножаем значения DC на калибровочный коэффициент,
    
– с помощью модифицированного закона Бугера-Ламберта-Бера (\ref{blb}) находим относительные коэффициенты поглощения ($\Delta \mu_a$),

– решаем линейную систему уравнений с двумя неизвестными (\ref{konz}),

– находим $\Delta C_{HbO_2}$ и $\Delta C_{HHb}$,

– вычисляем $\Delta C_{THb}$ по формуле \ref{ob}.

\begin{figure}[!h]
\begin{center}
\includegraphics[width=\textwidth]{схемочка.pdf}
\caption{\centering Схема реакции организма на стимул}
\label{45}
\end{center}
\end{figure}

Так как биоткань является неоднородной средой с большим количеством поглощающих веществ, ослабление интенсивности света происходит согласно модифицированному закону Бугера-Ламберта-Бера:
\begin{equation}\label{blb}
I=I_0\cdot e^{-\mu_a \cdot r \cdot DPF},
\end{equation}

где $I$ и $I_0$ -- 
это интенсивности выходящего и входящего из ткани излучения соответственно;

$\mu_a$ - коэффициент поглощения среды (1/см);

$r$ - толщина слоя вещества, через которое проходит свет (см);

$DPF$ - параметр дифференциального пробега фотона, учитывающий увеличение пути миграции фотонов за счет рассеяния.

$DPF$ зависит от пола, возраста и длины волны. Для этого эксперимента выбраны усредненные значения для $\lambda= 692$ нм $DPF = 6.51$, а для $\lambda= 834$ нм $DPF = 5.86$.
Коррекция закона с помощью DPF не так важна, поскольку для данных задач не нужна количественная оценка параметров, а достаточно определить присутствует ли активация и в каких именно областях она происходит.

Подставляя значения измеренных интенсивностей в моменты времени $t_i$ и $t_{i+1}$, находим $\Delta \mu_a$:
\begin{equation}\label{blb}
\begin{cases}
   {I_i=I_0 \cdot e^{-\mu_{a,i}\cdot r \cdot DPF}}\\
   {I_{i+1}=I_0\cdot e^{-\mu_{a,{i+1}}\cdot r \cdot DPF}}
 \end{cases}
 \Rightarrow
 \begin{cases}
   {ln{\frac{I_0}{I_i}}=\mu_{a,i} \cdot r \cdot DPF}\\
   {ln{\frac{I_0}{I_{i+1}}}=\mu_{a,{i+1}} \cdot r \cdot DPF},
 \end{cases}
\end{equation}
\vspace{-5mm}

где $I_i$ и $I_{i+1}$ - интенсивности в моменты времени $t_i$ и $t_{i+1}$ соответственно (шаг по времени равен 0.2 секунды); 

$\mu_{a,i}$ и $\mu_{a,{i+1}}$ - коэффициенты поглощения ткани в моменты времени $t_i$ и $t_{i+1}$ соответственно; 

OD - оптическая плотность исследуемой ткани.

\begin{equation}\label{blb}
\Delta OD=ln{\frac{I_0}{I_i}}-ln{\frac{I_0}{I_{i+1}}}=\Delta \mu_a \cdot r \cdot DPF.
\end{equation}

Формула 3.3 показывает зависимость изменения оптической плотности биоткани от относительного коэффициента поглощения. Соответствующие расстояния $r$ от источников до приемника для разных каналов представлены в таблице 3.1.

\begin{table}[!h]
\raggedright
\captionsetup[table]{singlelinecheck=false,justification=raggedleft}
%\ttabbox
{\caption{Расстояния от источников до приемника}}
{\begin{tabular}{|c|c|c|c|c|c|c|c|c|}
\hline
\multicolumn{9}{|c|}{Канал А (слева)}\\
\hline
Номер диода    &  1   &  2 &  3  &  4  &  5  & 6 &  7  &  8    \\
\hline
Расстояние, см &  1,95 &  2,43  &  2,98 &  3,47 &  2,02 & 2,51 & 3,03 & 3,54 \\
\hline
\multicolumn{9}{|c|}{Канал В (справа)}\\
\hline
Номер диода  & 1 &  2  &  3   &  4  & 5  &  6   &  7  &  8    \\
\hline
Расстояние, см & 1,96 & 2,48 & 2,95 & 3,46 &  2,02 &  2,54 &  3,05 & 3,52\\
\hline
\end{tabular}}
%\end{center}
\end{table}

Предположим, что в поглощении участвуют три основных хромофора: оксигемоглобин, дезоксигемоглобин и вода. Чтобы избавиться от третьего неизвестного в уравнении \ref{konz}, примем концентрацию воды 75\%.
\begin{equation}\label{konz}
\Delta \mu_a^{\lambda} =\sum{\varepsilon_i \cdot \Delta C_i},
\end{equation}

где $\Delta \mu_a^{\lambda}=\Delta \mu_{a,i+1}-\Delta \mu_{a,i}$ - изменение коэффициента поглощения за единицу времени (1/см);

$\varepsilon_i$ - молярный коэффициент экстинкции вещества на определённой длине волны в 1/(мМ * см);

$\Delta C_i$ - изменение концентрации вещества в единицу времени.

Примем, что для $\lambda= 692$ нм $DPF = 6.51$, а для $\lambda= 834$ нм $DPF = 5.86$.
Молярные коэффициенты экстинкции для $\lambda= 692$ нм равны $\varepsilon_{HHb}=4,7564$, $\varepsilon_{O2Hb}=0,9558$ и $\varepsilon_{H2O}=9,695 \cdot 10^{-8}$.
Молярные коэффициенты экстинкции для $\lambda= 834$ нм равны $\varepsilon_{HHb}=1,7891$, $\varepsilon_{O2Hb}=2,3671$ и $\varepsilon_{H2O}=60,6 \cdot 10^{-8}$.

Относительные концентрации общего гемоглобина вычисляются путем суммирования относительных концентраций окси- и дезоксигемоглобина:
\begin{equation}\label{ob}
\large \Delta C_{THb}=\Delta C_{HbO_2}+\Delta C_{HHb}.
\end{equation}

Все сигналы, зарегистрированные с устройства OxiplexTS были импортированы в MATLAB для дальнейшей обработки. Блок-схема этапов обработки сигналов, использованных в данном исследовании, и пример результатов каждого этапа показаны на рисунке . Проведено четыре основных обработки: исключение плохих каналов, удаление артефактов движения и нежелательных физиологических сигналов, преобразование данных в изменения концентрации оксигенированных гемоглобинов, дезоксигенированных и общих гемоглобинов и расчет уровня активности по гемодинамическому ответу.
На рисунках \ref{1}, \ref{2},  \ref{3}, \ref{4} показаны графики зависимости изменений относительных концентраций за время эксперимента для канала А (слева). Розовым цветом выделены промежутки времени, соответствующие стимуляции, белым - время, когда испытуемый находился в покое. Зависимости построены с помощью программного кода Python. На рисунке \ref{risB} приведены соответствующие данные для канала В (справа). Концентрация общего гемоглобина отражает кровенаполнение ткани, оксигемоглобина - поступление кислорода, а дезоксигемоглобина - его потребление.

\begin{figure}[!h]
\begin{center}
\begin{minipage}[h]{0.48\linewidth}
\includegraphics[width=\textwidth, keepaspectratio]{135.png}
\end{minipage}
\begin{minipage}[h]{0.48\linewidth}
\includegraphics[width=\textwidth, keepaspectratio]{graf98.png}
\end{minipage}
\caption{\centering \onehalfspacing {Графики изменения концентраций окси-, дезокси- и общего гемоглобина для расстояния <<источник-приемник>> равного 3,54 см}}
\label{1}
\end{center}
\end{figure}
\vspace{-5mm}

\begin{figure}[!h]
\begin{center}
\begin{minipage}[h]{0.48\linewidth}
\includegraphics[width=\textwidth, keepaspectratio]{graf27.png}
\end{minipage}
\begin{minipage}[h]{0.48\linewidth}
\includegraphics[width=\textwidth, keepaspectratio]{graf99.png}
\end{minipage}
\caption{\centering \onehalfspacing {Графики изменения концентраций окси-, дезокси- и общего гемоглобина для расстояния <<источник-приемник>> равного 3,03 см}}
\label{2}
\end{center}
\end{figure}

\begin{figure}[!h]
\begin{center}
\begin{minipage}[h]{0.48\linewidth}
\includegraphics[width=\textwidth, keepaspectratio]{graf29.png}
\end{minipage}
\begin{minipage}[h]{0.48\linewidth}
\includegraphics[width=\textwidth, keepaspectratio]{graf97.png}
\end{minipage}
\caption{\centering \onehalfspacing {Графики изменения концентраций окси-, дезокси- и общего гемоглобина для расстояния <<источник-приемник>> равного 2,51 см}}
\label{3}
\end{center}
\end{figure}
  
\begin{figure}[!h]
\begin{center}
\begin{minipage}[h]{0.48\linewidth}
\includegraphics[width=\textwidth, keepaspectratio]{graf31.png}
\end{minipage}
\begin{minipage}[h]{0.48\linewidth}
\includegraphics[width=\textwidth, keepaspectratio]{graf96.png}
\end{minipage}
\caption{\centering \onehalfspacing {Графики изменения концентраций окси-, дезокси- и общего гемоглобина для расстояния <<источник-приемник>>  равного 2,02 см}}
\label{4}
\end{center}
\end{figure}


\begin{figure}[!h]
\begin{center}

\begin{minipage}[h]{0.45\linewidth}
\includegraphics[width=\textwidth, keepaspectratio]{graf33}
\end{minipage}
%\hfill
\begin{minipage}[h]{0.45\linewidth}
\includegraphics[width=\textwidth, keepaspectratio]{graf34}
\end{minipage}

%\vfill

\begin{minipage}[!h]{0.45\linewidth}
\includegraphics[width=\textwidth, keepaspectratio]{graf35}
\end{minipage}
%\hfill
\begin{minipage}[h]{0.45\linewidth}
\includegraphics[width=\textwidth, keepaspectratio]{graf36}
\end{minipage}
\caption{\centering \onehalfspacing {Графики изменения концентраций окси-, дезокси- и общего гемоглобина на различных расстояниях «источник-приемник» для канала В (справа)}}
\label{risB}
\end{center}
\end{figure}

В ходе исследования проведено пять этапов обработки: исключение ненужных каналов, удаление артефактов движения и нежелательных физиологических сигналов, преобразование данных в изменения концентрации оксигенированных гемоглобинов, дезоксигенированных и общих гемоглобинов и расчет уровня активности по гемодинамическому ответу. Этапы обработки сигналов на примере одного сигнала и результаты каждого этапа показаны на рисунке \ref{алг}.
В первую очередь были исключены для обработки сигналы, снятые на расстояниях «источник-приемник» 2,02 и 2,51 см, так как с помощью них можно отследить только гемодинамику поверхностных слоев (кожа, тканевая оболочка головного мозга), функциональную же активность мозга нельзя визуализировать с помощью этих каналов. 
Далее обработка данных включала в себя удаление артефактов движения, дыхательных колебаний, исключение линейного тренда. 

Алгоритм обработки состоит из следующих этапов:

– удаление линейного тренда из экспериментальных данных,

– низкочастотная фильтрация сигнала для исключения физиологических шумов (кроме дыхания),

– вейвлет-преобразование для удаления артефактов движения,

– сглаживание сигнала.
\begin{figure}[!h]
\begin{center}
\includegraphics[width=\textwidth]{алгоритм.png}
\caption{\centering \onehalfspacing {Этапы обработки сигнала: удаление линейного тренда (левый верхний рисунок), низкочастотная фильтрация (правый верхний), вейвлет-преобразование (левый нижний) и сглаживание сигнала (правый нижний) }}
\label{алг}
\end{center}
\end{figure}

На рисунке \ref{card} представлена динамическая карта изменений окси-, дезокси и общего гемоглобина на разных глубинах зондирования. Ширина горизонтального слоя была выбрана в соответствии с таблицей \ref{глуб} (глубина зондирования рассчитывается как усредненное между 1/2-1/3 от расстояния «источник-приемник»). Из всех сигналов удаляется линейный тренд, чтобы было удобнее сопоставлять отклик на аудио стимул по разным слоям (рис. \ref{card}) без дополнительных искажений. 
\begin{figure}[!h]
\begin{center}
\includegraphics[width=0.75\textwidth]{карта.pdf}
\caption{\centering \onehalfspacing {Динамическая карта изменений общего гемоглобина и его фракций в зависимости от глубины зондирования}}
\label{card}
\end{center}
\end{figure}

\begin{table}[!h]
\raggedright
\captionsetup[table]{singlelinecheck=false,justification=raggedleft}
%\ttabbox{
\caption{\onehalfspacing Глубина зондирования в зависимости от расстояния <<источник-приемник>>}\label{глуб}
\begin{tabular}{|c|c|c|c|c|}
\hline
Расстояние <<источник-приемник>>, см & 2,02 & 2,51 & 3,03 & 3,54\\
\hline
Глубина зондирования, мм & \approx 8  & \approx 10 & \approx 13 & \approx 15\\
\hline
\end{tabular}
\end{table}
\newpage
По рисунку \ref{card} можно видеть, что распределение концентраций общего гемоглобина и его фракций различное по слоям. Динамика концентрации дезоксигенированного гемоглобина практически не изменяется с увеличением глубины. Различия по слоям оксигенированного и общего гемоглобинов проявляются сильнее.

Практически на всех зависимостях прослеживается четкая корреляция с аудиовоздействием: при стимуляции увеличивается концентрация оксигемоглобина и уменьшается концентрация дезоксигемоглобина, что говорит об усилении кровотока. В таких процессах всегда присутствует задержка на 2-5 с относительно начала стимуляции, но так как взят большой промежуток времени, на графиках это незаметно. Наилучший результат показало картирование на расстоянии 3,54 см. Здесь сильнее всего заметно нарастание $\Delta [HbO_2]$ и $\Delta [THb]$ и убывание $\Delta [HHb]$. На рис. \ref{4} видно, что при зондировании на расстоянии 2,02 см динамика концентраций уже не коррелируется с подачей стимула, так как такой глубины зондирования недостаточно, чтобы отследить вызванную нейронную активность головного мозга. На этом расстоянии видна только гемодинамика верхних слоев кожи головы и черепа. Поэтому для данной задачи расстояние от источника до приемника должно быть 2,51 см или выше, чтобы обеспечить картирование глубиной зондирования 1-1,5 см и мониторинг вызванной нейронной активности височной доли коры головного мозга.

По получившимся данным с правой стороны головы видно, что датчик канала В не попал в область слуховой коры и отклик на аудиосигнал менее выражен. Причиной этому может быть неточное позиционирование датчика и не попадание в зону слуховой коры. Также на графиках \ref{risB} присутствует дыхательная компонента (0,15-0,4 Гц). Ее можно выделить с помощью преобразования Фурье и устранить вейвлет-преобразованием. Распространенной причиной появления дыхательной компоненты является, попадание в область исследования магистрального венозного сосуда. Различия в сигналах с правой и левой стороны головы также подтверждают асимметричность левого и правого полушария, в частности, височных долей. Левая височная доля играет важную функциональную роль в восприятии звуковой информации и речевого материала, является доминантной. Правая височная доля имеет другие функции (воспринимает музыкальные звуки, обрабатывает информацию от зрительных источников и др.). Сигнал с правого полушария не коррелирует с вызванным стимулом, так как функции правой височной доли не были в полной мере задействованы в данном эксперименте. 
         
Волосяной покров, создающий воздушные прослойки между датчиком и кожей головы, зашумляет сигнал. Датчик должен плотно прилегать к поверхности головы, что было обеспечено с помощью разработанной конструкции (см. главу 2).

\newpage