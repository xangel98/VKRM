\newpage
\begin{center}
ЗАКЛЮЧЕНИЕ
\end{center}
\vspace{6mm}
%\refstepcounter{chapter}
\addcontentsline{toc}{chapter}{ЗАКЛЮЧЕНИЕ}


В ходе выпускной квалификационной работы бакалавра:

– изучены медицинские подходы, обеспечивающие контроль слухового восприятия,

– проанализированы все существующие методы контроля вызванной активности головного мозга,

– представлены общесистемные решения в виде биотехнической системы объективного контроля слухового восприятия,

– разработана конструкция БТС, обеспечивающая точное позиционирование спектрофотометрических датчиков,

– проведены исследования на разных глубинах зондирования, с целью определения наиболее подходящего,

– разработан алгоритм обработки экспериментальных данных.

Следующими возможными направлениями совершенствования и развития данной работы являются:

– увеличение количества каналов оптического датчика,
    
– уменьшение габаритов устройства путем усовершенствования аппаратной части,
   
– повышение качества обработки данных,
    
– cоздание удобного для пользователя интерфейса блока отображения информации,
   
– совмещение непрерывного и временного/частотного режима: использование  широкополосных  источников зондирующего излучения  совместно  с  несколькими  узкополосными амплитудно-модулированными  или импульсными источниками света разной длины волны, регистрируя сигнал несколькими приемниками.

Метод спектрофотометрии отличается от других существующих методов визуализации сосудистого русла тем, что он позволяет не только диагностировать состояние гемодинамики, но и оценивать эффективность утилизации кислорода тканями. Оценивая гемодинамику исследуемой области, можно выявить ряд патологий височной доли коры головного мозга, в частности, слухового центра.

Предложенная биотехническая система объективного контроля слухового восприятия спектрофотометрическим методом является доступной альтернативой другим системам объективного контроля, использующим другие подходы для контроля слухового восприятия.

