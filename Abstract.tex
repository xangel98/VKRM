\newpage
\begin{center}
РЕФЕРАТ
\end{center}
\vspace{6mm}
%\refstepcounter{chapter}
%\addcontentsline{toc}{chapter}{РЕФЕРАТ}

Расчетно-пояснительная записка 78 с., 24 рис., 4 табл., 30 источн. 

СЛУХОВОЕ ВОСПРИЯТИЕ, ГЕМОДИНАМИЧЕСКИЕ ПАРАМЕТРЫ, ОПТИЧЕСКИЕ СВОЙСТВА БИОЛОГИЧЕСКИХ ТКАНЕЙ, РАЗРАБОТКА КОНСТРУКЦИИ БЛОКА БТС, МЕТОД СПЕКТРОФОТОМЕТРИИ, ФУНКЦИОНАЛЬНАЯ АКТИВНОСТЬ ГОЛОВНОГО МОЗГА.

Работа включает в себя медико-биологическую, проектно-конструкторскую и исследовательскую части.

Объектом разработки является биотехническая система объективного контроля слухового восприятия спектрофотометрическим методом.

Предмет – блок БТС, обеспечивающий одновременную подачу звукового сигнала и работу спектрофотометрических датчиков в непрерывном режиме.

Цель работы – разработка оптимального конструкторского и схемотехнического решений, а также разработка алгоритма обработки экспериментальных данных, полученных при проведении измерений на различных расстояниях между источником и приемником.

Методология и методы исследования: физической основой методов спектрофотометрии является взаимодействие фотонов света с биологической тканью. Моделирование конструкции БТС осуществлено с помощью программного обеспечения Autodesk Inventor. 
Расчеты выполнены в среде MATLAB. Пост-обработка результатов моделирования и экспериментов выполнена с использованием программных кодов, реализованных на Python автором настоящей работы.

Выделены следующие задачи:

– разработка устройства, обеспечивающего одновременную подачу звукового сигнала и непрерывную работу спектрофотометрических датчиков,

– обеспечение надежной фиксации датчиков для устранения двигательных артефактов,

– разработка алгоритмов обработки сигналов со слуховой коры, записанных с помощью устройства OxiplexTS.










%В первой главе приведен обзор оптических свойств тканей головы и головного мозга и теоретических основ спектрофотометрического метода. Рассмотрены существующие методики контроля слухового восприятия, предложен алгоритм оценки вызванного гемодинамического отклика по оптическим параметрам, сделан обзор характеристик аудиометрического оборудования.

%Во второй главе работы приведено обоснование конструкции комбинированного датчика, требования к корпусу модуля, необходимые для обеспечения нужной глубины зондирования и чувствительности оптического канала. Показаны этапы разработки крепления датчиков к аудиосистеме, определены требования к характеристикам аудиооборудования. Также подобраны электрические компоненты для комбинированного датчика и предложен алгоритм обработки экспериментальных данных.

%Третья глава посвящена экспериментальному исследованию временной динамики изменения концентраций окси- и дезоксигемоглобина при зондировании области височной доли головного мозга, содержит анализ экспериментальных данных с точки зрения глубины зондирования тканей головного мозга, связанной с расстоянием между источником и приемником. Приведено сравнение получаемой динамики сигналов - концентраций окси- и дезокси- гемоглобина и сатурации.


