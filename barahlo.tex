\begin{equation}\label{blb}
\large{I=I_0\cdot e^{-\mu_a \cdot r \cdot DPF}},
\end{equation}

где $I$ и $I_0$ -- 
это интенсивности выходящего и входящего из ткани излучения соответственно, $\mu_a$ - коэффициент поглощения среды (1/см),  $r$ - толщина слоя вещества, через которое проходит свет (см), DPF - параметр дифференциального пробега фотона, учитывающий увеличение пути миграции фотонов за счет рассеяния. \\

\begin{equation}\label{blb}
\begin{cases}
   {I_i=I_0 \cdot e^{-\mu_{a,i}\cdot r \cdot DPF}}\\
   {I_{i+1}=I_0\cdot e^{-\mu_{a,{i+1}}\cdot r \cdot DPF}}
 \end{cases}
 \Rightarrow
 \begin{cases}
   {\frac{I_0}{I_i}=e^{ \mu_{a,i}\cdot r \cdot DPF}}\\
   {\frac{I_0}{I_{i+1}}=e^{ \mu_{a,{i+1}}\cdot r \cdot DPF}}
 \end{cases}

 \Rightarrow
 %\Rightarrow
 \begin{cases}
   {ln{\frac{I_0}{I_i}}=\mu_{a,i}\cdot r\cdot DPF}\\
   {ln{\frac{I_0}{I_{i+1}}}=\mu_{a,{i+1}}\cdot r \cdot DPF}
 \end{cases}
\end{equation}

где $I_i$ и $I_{i+1}$ - интенсивности в моменты времени $t_i$ и $t_{i+1}$ соответственно (шаг по времени равен 0.2 секунды), $\mu_{a,i}$ и $\mu_{a,{i+1}}$ - коэффициенты поглощения ткани в моменты времени $t_i$ и $t_{i+1}$ соответственно, OD - оптическая плотность исследуемой ткани.

\begin{equation}\label{blb}
\Delta OD=ln{\frac{I_0}{I_i}}-ln{\frac{I_0}{I_{i+1}}}=\Delta \mu_a \cdot r \cdot DPF
\end{equation}

\begin{equation}\label{blb}
\Delta \mu_a^{\lambda} =\sum{\varepsilon_i \cdot \Delta C_i},
\end{equation}

где $\Delta \mu_a^{\lambda}=\Delta \mu_{a,i+1}-\Delta \mu_{a,i}$ - изменение коэффициента поглощения за единицу времени (1/см), $\varepsilon_i$ - молярный коэффициент экстинкции вещества на определённой длине волны в 1/(мМ * см), $\Delta C_i$ - изменение концентрации вещества в единицу времени.

Примем, что для $\lambda= 692$ нм $DPF = 6.51$, а для $\lambda= 834$ нм $DPF = 5.86$.
Молярные коэффициенты экстинкции для $\lambda= 692$ нм равны $\varepsilon_{HHb}=4,7564$, $\varepsilon_{O2Hb}=0,9558$ и $\varepsilon_{H2O}=9,695 \cdot 10^{-8}$.
Молярные коэффициенты экстинкции для $\lambda= 834$ нм равны $\varepsilon_{HHb}=1,7891$, $\varepsilon_{O2Hb}=2,3671$ и $\varepsilon_{H2O}=60,6 \cdot 10^{-8}$.


\begin{table}[!h]
\caption{\label{tab:2}\\Режимы облучения биоткани: характеристики и основные параметры, измеряемые с помощью этих режимов}
\begin{tabular}{|p{7.5cm}|p{4cm}|p{4cm}|}
\hline
Характеристики & Непрерывный режим облучения & Модулированный режим облучения \\
\hline
Частота дискретизации, Гц & \leq 50 & 0-50 \\
\hline
Пространственное разрешение, см &\leq 1 & \leq 1\\
\hline
Разделение сигналов в головном мозге и в ткани головного мозга & Невозможно & Возможно\\
\hline
Возможность измерения глубинных структур мозга & Только на новорожденных & Только на новорожденных\\
\hline
Размер устройства & Небольшой размер & Крупногабаритные устройства\\
\hline
Мобильность & Легко сделать мобильным & Трудно сделать мобильным\\
\hline
Стоимость & Низкая & Высокая \\
\hline
\multicolumn{3}{|c|}{Измеряемые параметры} \\
\hline
[O_2Hb], [HHb], [tHb] & Да, относительные значения & Да, абсолютные значения \\
\hline
Измерение коэффициента рассеяния и поглощения и длины пути фотонов & Нет & Да \\
\hline
Измерение насыщения ткани кислородом (сатурации), \% & Нет & Да \\
\hline
\end{tabular}
\end{table}
В эксперимент проводился при помощи спектофотометрического прибора — тканевого оксиметра «OxiplexTS», калибровочных блоков