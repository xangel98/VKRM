\newpage
\begin{center}
ВВЕДЕНИЕ
\end{center}
\vspace{6mm}
%\refstepcounter{chapter}
\addcontentsline{toc}{chapter}{ВВЕДЕНИЕ}

В последнее время одной из основных тенденций в хирургии стала интраоперационная роботоассистенция, позволяющая повысить безопасность операции, увеличить точность проведения хирургического вмешательства за счет интраоперационной навигации. Такие системы востребованы, в том числе, и в эндоваскулярной хирургии, устраняющей нарушения системы кровообращения, такие как
артериальная непроходимость или аневризмы. Все это позволяет выполнять хирургические манипуляции с большей точностью и значительно быстрее по сравнению с традиционными методами.

Роботизированные ультразвуковые системы применяются при операциях на магистральных артериях для отслеживания малоинвазивных хирургических процедур. Автоматизация ультразвуковой визуализации с помощью роботов-манипуляторов во время хирургического вмешательства помогает разгрузить врачей-сонографистов. Задача роботизированного трекинга хирургического инструмента до сих пор остается актуальной из-за необходимости повышать качество изображения и угла поля зрения, тем самым улучшая существующие интервенционные методы визуального контроля для осуществления новых малоинвазивных хирургических процедур.


\newpage

