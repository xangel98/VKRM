\newpage\newpage
\begin{center}
ПЕРЕЧЕНЬ ОБОЗНАЧЕНИЙ И СОКРАЩЕНИЙ
\end{center}
%\refstepcounter{chapter}
\vspace{6mm}
%\addcontentsline{toc}{chapter}{ПЕРЕЧЕНЬ ОБОЗНАЧЕНИЙ И СОКРАЩЕНИЙ}

В данном отчете о ВКРБ применяют следующие сокращения с соответствующими определениями.
\renewcommand{\arraystretch}{1.3}
\begin{table}[!h]
\flushleft
\begin{tabular}{l l}

К-БИКС &- красная и ближняя инфракрасная спектроскопия \\
БТС &- биотехническая система\\
ЦНС &- центральная нервная система\\
ЭЭГ &- электроэнцефалография \\
КТ &- компьютерная томография \\
ПЭТ &- позитронно-эмиссионная томография \\
МРТ &- магнитно-резонансная томография \\
ТПИ & - теория переноса излучения \\
УФ  &- ультрафиолетовый \\
ИК &- инфракрасный \\
СФ &- спектрофотометрия \\
CW &- непрерывный режим \\
%AC &- амплитуда волны интенсивности \\
%DC &- средняя по времени интенсивность \\
БИК &- ближний инфракрасный \\
ОФЭКТ &- однофотонная эмиссионная компьютерная томография \\
фМРТ &- функциональная магнитно-резонансная томография \\
rCBF &- региональный мозговой кровоток \\
fNIRS &- функциональная ближняя инфракрасная спектроскопия\\
ИГ &- извилина Гешля \\
СБ &- сильвиева борозда \\
ВВИ &- верхняя височная извилина \\
СКП &- слуховой кортикальный пояс \\ 
ЛД &- лазерный диод \\
ПК &- персональный компьютер \\
ШИМ &- широтно-импульсная модуляция \\ 
ЦАП &- цифро-аналоговый преобразователь\\
PGA &- усилитель с программированием коэффициента усиления \\
АЦП &- аналого-цифровой преобразователь \\
\end{tabular}
\label{tab:my_label}
\end{table}
